\documentclass[a4paper]{article}

\usepackage[francais]{babel}
\usepackage[utf8x]{inputenc}
\usepackage[T1]{fontenc}
%\usepackage{amsmath}
\usepackage{graphicx}
\usepackage[colorinlistoftodos]{todonotes}
\usepackage{comment}
\usepackage{float}
\usepackage[colorlinks=true,linkcolor=black,citecolor=black]{hyperref}
\usepackage{colortbl}


\title{\bsc{INSA} de Rennes \\ Département Informatique \\ \bigskip \hrule \bigskip ArchiPoilus \\ \bigskip Système d'annotation et de navigation dans des images d'archives \\ \bigskip Dossier de planification initiale \bigskip \hrule}
\author{Raphaël \bsc{Baron}, Pierre-Olivier \bsc{Bouteau}, Nicolas \bsc{Charpentier}, \\ Clément \bsc{Leboullenger}, Thomas \bsc{François}, Benoit \bsc{Travers}}

\begin{document}

\maketitle
\thispagestyle{empty}

\newpage
\tableofcontents
\thispagestyle{empty}

\newpage
\phantomsection
\addcontentsline{toc}{section}{Introduction}
\section*{Introduction}

	Les Archives départementales d'Ille-et-Vilaine sont en charge de la collecte, du classement, de la conservation et de la communication des archives qui constituent une grande partie du patrimoine historique écrit du département. 
\`A l'approche du centenaire de la Première Guerre Mondiale, les Archives, en collaboration avec l'\bsc{INSA} de Rennes, ont choisi de proposer aux étudiants en quatrième année du département informatique un sujet de mise en valeur des documents li\'es \`a la Grande Guerre.

	L'objectif du projet est de construire un outil d'annotations semi-auto\-matiques des documents liés au recrutement militaire concernant cette période. L'outil doit permettre à un utilisateur, visiteur des archives, d'associer lecture et annotation du document avec l'image numérisée qui lui correspond. En plus de naviguer et d'annoter les images, l'usager peut utiliser ces annotations pour rechercher des documents, par exemple trouver une personne grâce a son nom.\\
	
	Ce Dossier de planification initiale présente notre projet dans les grandes lignes ainsi que la planifications des tâches à réaliser tout en indiquant les dates consernant les livrables et les soutenances. Il contiendra notamment un premier diagramme de Gantt résumant graphiquement les intéractions et recouvrements temporels des tâches préalablement définies.\\
	
	Ce travail a fait l'objet d'échanges réguliers avec M. Jean-Yves Le Clerc, adjoint au directeur des Archives départementales d’Ille-et-Vilaine, afin de cibler au mieux les besoins auxquels devra répondre notre produit, et Me. Karen Février, chef de projet chez ATOS, qui nous offre un suivi sur l'aspect gestion de projet ainsi que sur la rédaction des documents.

\newpage

\section{Présentation du projet : ArchiPoilus}

\subsection{Sujet - Cahier des charges}

\subsection{Contexte}

\subsection{Fonctionnalités Générales}

	Les quatres principales fonctionnalités sont : la recherche, la navigation, la visualisation des documents et l'annotation de ces dernier. Chacune de ses fonctionnalités est nécessaire ...

\subsubsection{Recherche}
\subsubsection{Navigation}
\subsubsection{Visualisation des documents}
\subsubsection{Annotation}

\subsubsection{Authentification}

	Cette partie sera moins prioritaire que les fonctionnalités précédantes. En effet, notre premier objectif est de réaliser une première version fonctionnelle, c'est-à-dire uniquement avec les fonctionnalités de base.

\subsubsection{Administration}

	Cette fonctionnalité est la moins prioritaire.
	
\section{Planification des tâches}
(Présentation des deadlines à respecter, du temps de travail par semaine, des tâches moins importantes, ...)
	\subsection{Dates des Soutenances et des Livrables}
	\subsection{Cycle en V}
	Justification par rapport à une méthode agile trop dure à mettre en place car trop peut de temps pour faire itération et rendu des rapport correspond à un cycle en V
	\subsection{Planification des tâches}
	(travail sous ms project : Voir avec la nvelle version si une autre visualisation de la planif n'est pas plus intéressante, temps de travail par semaine, semaines particulière)

\section{Gestion de projet}

\newpage
\phantomsection
\addcontentsline{toc}{section}{Conclusion}
\section*{Conclusion}

TODO.

\end{document}
