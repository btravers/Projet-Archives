\documentclass[a4paper]{article}

\usepackage[francais]{babel}
\usepackage[utf8x]{inputenc}
\usepackage[T1]{fontenc}
\usepackage{graphicx}
\usepackage{float}
\usepackage{colortbl}

\title{\bsc{INSA} de Rennes \\ Département Informatique \\ \bigskip \hrule \bigskip ArchiPoilus \\ \bigskip Système d'annotation et de navigation dans des images d'archives \\ \bigskip Rapport de spécification fonctionnelle \bigskip \hrule}
\author{Raphaël \bsc{Baron}, Pierre-Olivier \bsc{Bouteau}, Nicolas \bsc{Charpentier}, \\ Clément \bsc{Leboullenger}}

\begin{document}

\maketitle
\thispagestyle{empty}

\newpage
\tableofcontents
\thispagestyle{empty}

\newpage
\phantomsection
\addcontentsline{toc}{section}{Introduction}
\section*{Introduction}

\section{Modèle Conceptuel de Données}

\begin{table}[H]
  \centering
   \small
	\begin{tabular}{|c|p{12cm}|}
   		\hline
   			\rowcolor{lightgray}\multicolumn{2}{|c|}{\textbf{Table - Utilisateur}} \\
   		\hline
   			\multicolumn{2}{|l|}{Attributs} \\
   		\hline
			identifiant & (Primary key, String) clé unique représentant le pseudonyme de l'utilisateur \\
			mot_de_passe & (String) mot de passe crypté pour permettre l'authentification de l'utilisateur \\
			email & (String) email de l'utilisateur \\
			statut & (Int, défaut : 1) détermine les droits d'accès de l'utilisateur \\
   		\hline
 	\end{tabular}
  \caption{Table - Utilisateur}
  \normalsize
  \label{tab: table_utilisateur}
\end{table}

\begin{table}[H]
  \centering
   \small
	\begin{tabular}{|c|p{12cm}|}
   		\hline
   			\rowcolor{lightgray}\multicolumn{2}{|c|}{\textbf{Table - Registre}} \\
   		\hline
   			\multicolumn{2}{|l|}{Attributs} \\
   		\hline
			identifiant & (Primary key, Int) clé unique \\
			commune & String commune du registre \\
			année & Int année du registre \\
			volume & Int numéro du volume du registre \\
   		\hline
 	\end{tabular}
  \caption{Table - Registre}
  \normalsize
  \label{tab: table_registre}
\end{table}


\begin{table}[H]
  \centering
   \small
	\begin{tabular}{|c|p{12cm}|}
   		\hline
   			\rowcolor{lightgray}\multicolumn{2}{|c|}{\textbf{Table - Table}} \\
   		\hline
   			\multicolumn{2}{|l|}{Attributs} \\
   		\hline
			identifiant & Int clé unique \\
			numéro & Int numéro de la table par rapport au registre \\
		\hline
 	\end{tabular}
  \caption{Table - Utilisateur}
  \normalsize
  \label{tab: table_utilisateur}
\end{table}

\begin{table}[H]
  \centering
   \small
	\begin{tabular}{|c|p{12cm}|}
   		\hline
   			\rowcolor{lightgray}\multicolumn{2}{|c|}{\textbf{Table - Fiche}} \\
   		\hline
   			\multicolumn{2}{|l|}{Attributs} \\
   		\hline
			identifiant & Int clé unique \\
			numéro & Int numéro de la fiche par rapport au registre \\
   		\hline
 	\end{tabular}
  \caption{Table - Fiche}
  \normalsize
  \label{tab: table_fiche}
\end{table}


\newpage
\phantomsection
\addcontentsline{toc}{section}{Conclusion}
\section*{Conclusion}


\appendix

\end{document}
